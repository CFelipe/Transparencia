\documentclass[12pt, a4paper]{article}
\usepackage[utf8]{inputenc}
\usepackage[brazilian]{babel} % Hifenização e dicionário
\usepackage[left=3.00cm, right=2.00cm, top=3.00cm, bottom=2.00cm]{geometry}

\begin{document}

    \begin{center}
      \textsc{UFRN --- Universidade Federal do Rio Grande do Norte} \\
      \textsc{DIMAp --- Departamento de Informática e Matemática Aplicada} \\
    \end{center}

    \bigskip

    \begin{tabular}{@{}ll@{}}
        \emph{Disciplina:} & DIM0433 --- Engenharia de Software \\
        \emph{Docente:}    & Márcia Jacyntha Nunes Rodrigues Lucena \\
        \emph{Discentes:}  & Felipe Cortez de Sá \small{(2012912357)} \\
                           & Graco Babeuf Vieira Silva \small{(2012912393)} \\
                           & Íslame Felipe da Costa Fernandes \small{(2012912437)} \\
                           & Rubens Viana Araújo \small{(2012912571)} \\
                           & Victor Almeida Schinaider \small{(2012912624)}
    \end{tabular}

    \bigskip

    \begin{center}
      \Large\textbf{Documento de Visão}
    \end{center}

    \section{Introdução}
        \subsection{Finalidade}
        \subsection{Escopo do documento}
        \subsection{Definições, acrônimos e abreviações}
        \subsection{Referências}

    \section{Contextualização}
        \subsection{Descrição do problema}
            \begin{tabular}{ p{4cm} | p{11cm} }
                \hline
                \textbf{Problemas} &
                Os problemas vão aqui \\ \hline
                \textbf{Pessoas atingidas} &
                As pessoas atingidas vão aqui \\ \hline
                \textbf{Cujo impacto é} &
                O impacto vai aqui \\ \hline
                \textbf{Uma solução bem sucedida traria} &
                O que a solução traria vai aqui \\ \hline
            \end{tabular}

    \subsection{Sentença de posição do produto}
    % Ou essa tabela está meio estranha, ou não está renderizando direito aqui.
    % Vou tentar ver em outro computador depois

    \section{Descrição dos stakeholders e dos usuários}
        \subsection{Principais stakeholders e usuários}
            \begin{tabular}{ p{5cm} | p{5cm} | p{5cm} }
                \hline
                \textbf{Identificação} &
                \textbf{Responsabilidades} &
                \textbf{Stakeholders} \\ \hline
                Gerentes do projeto & & \\ \hline
                Analistas de requisitos & & \\ \hline
                Arquiteto do projeto & & \\ \hline
                Projetista de interfaces do projeto & & \\ \hline
                Programadores & & \\ \hline
                Organização & & \\ \hline
                Usuário & & \\
            \end{tabular}
        \subsection{Necessidades chave dos stakeholders e dos usuários}
            \begin{tabular}{ p{3.75cm} | p{3.75cm} | p{3.75cm} | p{3.75cm} }
                \hline
                \textbf{No.} &
                \textbf{Descrição} &
                \textbf{Prioridade do cliente} &
                \textbf{Observações} \\ \hline
                & 1 & & \\ \hline
                & 2 & & \\ \hline
                & 3 & & \\ \hline
                & 4 & & \\ \hline
                & 5 & & \\ \hline
            \end{tabular}
        \subsection{Necessidades chave dos stakeholders e dos usuários}

    \section{Visão geral do produto}
        \subsection{Perspectiva do produto}
            \begin{tabular}{ p{7.5cm} | p{7.5cm} }
                \hline
                \textbf{Necessidades} &
                \textbf{Funcionalidades correspondentes} \\ \hline
                1. & \\ \hline
                2. & \\ \hline
                3. & \\ \hline
                4. & \\ \hline
                5. & \\ \hline
            \end{tabular}
        \subsection{Premissas e dependências}
        \subsection{Limites do produto}

    \section{Requisitos funcionais do produto}

    \section{Precedência e prioridades}
            \begin{tabular}{ p{3.75cm} | p{3.75cm} | p{3.75cm} | p{3.75cm} }
                \hline
                \textbf{No.} &
                \textbf{Funcionalidade} &
                \textbf{Prioridade do cliente} &
                \textbf{Entrega} \\ \hline
                & 1 & & \\ \hline
                & 2 & & \\ \hline
                & 3 & & \\ \hline
                & 4 & & \\ \hline
                & 5 & & \\ \hline
            \end{tabular}

    \section{Requisitos não-funcionais do produto}

    \section{Restrições técnicas}

\end{document}
