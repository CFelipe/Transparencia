\documentclass[12pt, a4paper]{article}
\usepackage[utf8]{inputenc}
\usepackage[brazilian]{babel} % Hifenização e dicionário
\usepackage[left=3.00cm, right=2.00cm, top=3.00cm, bottom=2.00cm]{geometry}
\usepackage{enumitem} % Para itemsep etc
\usepackage{tabu} % Para melhor criação de tabelas
\usepackage{color}
\usepackage{parskip} % Linha em branco entre parágrafos em vez de recuo

\tabulinesep=0.75ex % Espaçamento interno das tabelas

\begin{document}

    \begin{titlepage}
        \flushright
        \rule{\textwidth}{1pt}
        {\large \textsc{UFRN --- Universidade Federal do Rio Grande do Norte}
        \vspace{-1ex}}
        \rule{\textwidth}{1pt}

        \vfill

        {\Huge Sistema de Transparência Pública para a Assembleia Legislativa
        do RN \\[1ex] \LARGE Documento de Visão \\[2ex] \large Versão 1.4}

        \vfill

    \end{titlepage}

    \begin{tabu}{X | X[-1]}
        \hline
        Sistema de Transparência Pública para a Assembleia Legislativa do RN &
        UFRN \\ \hline
        Documento de Visão &
        Versão 1.4\\ \hline
    \end{tabu}

    \bigskip

    {\Large\textbf{Histórico de Revisão}}

    \begin{tabu}{X[-1] | X[-1, r] | X | X[-1]}
        \hline
        \textbf{Data} &
        \textbf{Versão} &
        \textbf{Descrição} &
        \textbf{Autor} \\ \hline
        03/04/2014 &
        1.0 &
        Concepção do documento &
        Íslame Felipe \\ \hline
        05/04/2014 &
        1.1 &
        Leitura do documento com adição de informações e comentários &
        Felipe Cortez \\ \hline
        07/04/2014 &
        1.2 &
        Preenchimento inicial das tabelas de stakeholders e usuários,
        necessidades chave, perspectiva do produto e precedências e prioridades
        &
        Felipe Cortez \\ \hline
        07/04/2014 &
        1.3 &
        Formatação de capa e correção de erros &
        Felipe Cortez \\ \hline
         08/04/2014 &
        1.4 &
        Definições, acrônimos, abreviações, referências, requisitos e funcionalidades &
        Íslame Felipe \\ \hline
    \end{tabu}

    \tableofcontents

    % \begin{tabular}{@{}ll@{}}
    %     \emph{Disciplina:} & DIM0433 --- Engenharia de Software \\
    %     \emph{Docente:}    & Márcia Jacyntha Nunes Rodrigues Lucena \\
    %     \emph{Discentes:}  & Felipe Cortez de Sá \small{(2012912357)} \\
    %                        & Graco Babeuf Vieira Silva \small{(2012912393)} \\
    %                        & Íslame Felipe da Costa Fernandes \small{(2012912437)} \\
    %                        & Rubens Viana Araújo \small{(2012912571)} \\
    %                        & Victor Almeida Schinaider \small{(2012912624)}
    % \end{tabular}
    %
    % \bigskip

    \newpage

    \begin{center}
      \Large\textbf{Documento de Visão}
    \end{center}

    \section{Introdução}
        \subsection{Finalidade}
        A finalidade deste documento é proporcionar uma perspectiva geral do
        projeto proposto, a fim de que seja possível compreender o seu produto.
        Assim, será possível entender as necessidades do sistema, as
        funcionalidades a serem implementadas e as restrições requisitadas.
        Este documento também visa entender o contexto da aplicação, seus
        stakeholders, requisitos, bem como desafios que devem ser enfrentados.

        \subsection{Escopo do documento}
        Este documento de visão se aplica ao Portal da Transparência da Assembleia
        Legislativa do Rio Grande do Norte. Este projeto propõe desenvolver um sistema
        web (na forma de portal) que deverá disponibilizar, de uma maneira transparente
        e imparcial, os aspectos que caracterizam os trâmites políticos,
        administrativos e econômicos da Assembleia Legislativa do Rio Grande do Norte.
        Como exemplos de tais aspectos estão a atuação dos parlamentares nas seções e
        nas votações de projetos importantes para a sociedade, bem como os projetos de
        leis que estão em votação atualmente. Eis o principal problema que deve ser
        resolvido: coletar informações concretas de tais aspectos e disponibilizá-las
        em um portal prático, isto é, de fácil manuseio e de pesquisa rápida, de modo a
        atrair o cidadão para que o mesmo possa usufruir do seu direito à informação.

        {\color{red} Colocar diagramas de contexto}

        \subsection{Definições, acrônimos e abreviações}
         \begin{tabu}{X | X}
                \hline
                \textbf{Palavra-chave} &
                \textbf{Definição}
                \\ \hline
                Parlamento &
                Constitucionalmente, e segundo o princípio da divisão dos poderes, constitui a sede do poder legislativo. Será representado pela Assembleia Legislativa do RN.
                \\ \hline
                Parlamentar & indivíduo que compõe o parlamento. Pela Constituição vigente (de 1988), os parlamentares devem ser eleitos por eleições livres e diretas e têm a função de elaborar as leis de regem o Estado.
                \\ \hline
Trâmites Legislativos & termo de alto nível usado neste documento para referir-se a toda e qualquer atividade desempenhada no âmbito do poder legislativ, seja fora ou dentro de plenário. O termo abrange discussões e votações de projetos, liberações de verbas, debates em comissões, reuniões e etc. 
                \\ \hline
Projeto & proposta, planejamento ou ideia que é escrita, analisada e votada em plenário, de acordo com o regimento interno da Assembleia e com a ordem de pautas. Pode referir-se a propostas de leis, de liberação de verbas, ou relacionado a assuntos de interesse geral da sociedade. 
                \\ \hline
Assessor & indivíduo que presta serviço de assessoria individualmente ao parlamentar ou à Assembleia enquanto instituição. 
                \\ \hline
Seção & reunião ordinária ou extraordinária dos parlamentares no plenário da Assembleia. Nas seções ordinárias, são debatidos assuntos importantes, analisados projetos ou realizadas votações. Nas seções extraordinárias são solicitadas, geralmente, para homenagens, condecorações, ou votações urgentes. A presença em seções é a atividade parlamentar mais básica.
                \\ \hline

Comissões & reuniões provisórias que reuni um subconjunto de parlamentares que devem analisar, investigar ou votar uma questão particular. A CPI (Comissão Parlamentar de Inquérito) é um exemplo de comissão que tem poder de investigação, a partir de escuta popular, de testemunhas ou de documentos. 
                \\ \hline
Orçamento público & é um instrumento de planejamento e execução das finanças públicas. Busca prevê as despesas e receitas públicas. Deve ser votada pelo parlamento, portanto tem caráter de documento legal e é aprovado por lei. 
                \\ \hline
            \end{tabu}
        

        \subsection{Referências}
        {\color{red} Este documento poderá ser complementado ao ter em mãos os
        documentos descritos}

    \section{Contextualização}
        \subsection{Descrição do problema}
            \begin{tabu}{X[1] | X[3]}
                \hline

                \textbf{Problemas} &
                \begin{minipage}[t]{\linewidth}
                \begin{itemize}[itemsep=.5ex,parsep=.0ex,after=\strut]
                \item Coletar informações acerca dos fatos administrativos,
                    políticos e econômicos da Assembleia Legislativa do RN
                \item Garantir a veracidade e integridade destas informações
                \item Fazê-las chegar aos cidadãos, por um portal de fácil acesso
                    e manuseio
                \item Instigar o cidadão a fiscalizar os trâmites legislativos
                    (o que lhe é de direito) através do uso da web
                \item Aproximar a Assembleia Legislativa e seus representantes
                    da população
                \end{itemize}
                \end{minipage}
                \\ \hline

                \textbf{Pessoas atingidas} &
                O cidadão é o principal agente atingido por este projeto, uma
                vez que o mesmo poderá inteirar-se acerca do que acontece na
                Assembleia, bem como das ações tomadas por parte dos seus
                representantes legislativos. A Assembleia Legislativa também
                deve ser beneficiada, pois terá a oportunidade de manter sua
                imagem transparente junto à população.
                \\ \hline

                \textbf{Cujo impacto é} &
                O principal impacto é o aprimoramento da visão crítica do
                cidadão, que poderá melhor discernir sobre suas próprias
                decisões no âmbito do seu exercício da democracia. A sociedade
                como um todo deve, portanto, ser beneficiada. A preocupação com
                a usabilidade facilitaria o acesso às informações.  Outro
                impacto, de certa forma, será a revelação de possíveis
                irregularidades que possam acontecer pelo mau uso do sistema
                representativo.
                \\ \hline

                \textbf{Uma solução bem \newline sucedida traria} &
                Traria as informações mais relevantes à população, de modo
                claro, objetivo e imparcial. Traria também um portal de fácil
                manuseio e que coloca tais informações em primeiro plano.
                \\ \hline

            \end{tabu}

        \subsection{Sentença de posição do produto}
            \begin{tabu}{X[1] | X[3]}
                \hline
                \textbf{Para} &
                A sociedade em geral
                \\ \hline
                \textbf{Quem} &
                Por meio da equipe de desenvolvimento
                \\ \hline
                \textbf{O} &
                É um portal para a transparência da Assembleia Legislativa do RN
                \\ \hline
                \textbf{Que} &
                Proporcionar amplo acesso às informações acerca dos trâmites
                legislativos
                \\ \hline
                \textbf{Diferente de} &
                Não se aplica
                \\ \hline
                \textbf{Nosso produto} &
                Traz a informação de maneira clara, objetiva e imparcial
                \\ \hline
            \end{tabu}

    \section{Descrição dos stakeholders e dos usuários}
        \subsection{Principais stakeholders e usuários}

            \begin{tabu}{X | X | X}
                \hline
                \textbf{Identificação} &
                \textbf{Responsabilidades} &
                \textbf{Stakeholders}
                \\ \hline
                Gerentes do projeto & Gerenciar o acompanhamento do projeto.
Avaliar a evolução do sistema. &
                Bernardo Gurgel
                \\ \hline
                Analistas de requisitos & Analisar os requisitos do sistema e do usuário.
Realizar a verificação (corretude e coerência).
Coordenar o processo de elicitação e especificação.&
                Íslame Felipe
                \\ \hline
                Arquiteto do projeto & Arquitetar o sistema. Propor a melhor forma de organizar as estruturas do sistema (modelo de arquitetura).&
                Todos
                \\ \hline
                Projetista de interfaces do projeto & Desenvolver e atualizar o projeto de interface de usuário. &
                Felipe Cortez
                \\ \hline
                Programadores & Desenvolver a codificação do sistema. Manter a integridade e evolução do código. Trabalho fundamentalmente em equipe. &
                Todos
                \\ \hline
                Organização & Prover a organização do projeto. &
                Ricardo Wagner, Rubens Viana e Victor Schinaider
                \\ \hline
                Usuário &Utilizar os incrementos do sistema. Emitir feedbacks sobre novos requisitos.
 &
                População geral
                \\ \hline
            \end{tabu}
        \subsection{Necessidades chave dos stakeholders e dos usuários}
            \begin{tabu}{X[-1, r] | X | X[r] | X}
                \hline
                \textbf{No.} &
                \textbf{Descrição} &
                \textbf{Prioridade do cliente} &
                \textbf{Observações} \\ \hline
                1 &
                Conhecer os parlamentares, seus partidos, coligações e ideologias. Monitorar a atuação dos parlamentares e seu envolvimento nas atividades legislativas. 
                & 1 & \\ \hline
                2 &
                Formar uma opinião pessoal acerca da atuação do parlamentar a fim de poder atuar da melhor maneira possível no processo eleitoral.
 &
                1 & \\ \hline
                3 &
               Conhecer os projetos eminentes, em pauta ou em espera pela pauta. Ter consciência da forma como os projetos são revertidos em benefícios à sociedade. &

                2 & \\ \hline
                4 &
              Conhecer o funcionamento da máquina pública. Poder acompanhar o direcionamento do dinheiro público. &
                2 & \\ \hline
                5 &
                Conhecer o funcionamento da Assembleia Legislativa do ponto de vista institucional. Aproximar-se do parlamento.
                &
                3 &
                \\ \hline
            \end{tabu}

    \section{Visão geral do produto}
        \subsection{Perspectiva do produto}
           
            \begin{tabu}{X[-1, r]| X}
                \hline
                \textbf{Necessidades} &
                \textbf{Funcionalidades correspondentes} \\ \hline
                1
                & Página detalhada do perfil do     parlamentar \\ \hline
               1, 2 & Busca por seções plenárias. Listar pauta, ata, resumo, data, parlamentares presentes e etc. \\ \hline
               3, 4,  & Páginas que detalham comissões em andamento \\ \hline
                1, 2 & Página especial que associa o parlamentar a atividades desenvolvidas fora a assembleia.\\ \hline
                1, 2 & Mostrar detalhes de cada processo sofrido pelo parlamentar. \\ \hline
                1, 2 & Pesquisa histórico de processos do parlamentar. \\ \hline
                1, 2, 4, 5 & Listar a prestação de conta mensal da assembleia\\ \hline
                1, 2, 4 & Página que detalha o gasto mensal do parlamentar \\ \hline
                1, 2 & Seção especial para a prestação de contas do parlamentar à justiça eleitoral \\ \hline
                2, 3, 4, 5, & Página que detalha as informações do orçamento público \\ \hline  
                2, 3, 4, 5 & Download do texto na íntegra do projeto do orçamento \\ \hline
                3, 4 & Pesquisar projetos por categoria/natureza, por data, por status ou tema. \\ \hline
                3, 4 & Página que detalha projetos.\\ \hline
                2, 3, 4 & Página que noticia comissões, seções, reuniões e ouros assuntos.\\ \hline
                1, 2  & Pesquisar histórico de ex-parlamentar \\ \hline
                1, 2 &Página de contato do cidadão com o cliente\\ \hline

                
                
            \end{tabu}
        \subsection{Premissas e dependências}
        O sistema é um portal web. Inicialmente deve-se desenvolvê-lo para
        acesso em qualquer navegador de um computador ligado à Internet.
        Posteriormente, pode-se pensar em abranger esta ideia para dispositivos
        móveis, ou até mesmo uma aplicação móvel.
        Uma dependência forte do sistema refere-se à sua alimentação. De fato, as informações devem, de alguma fonte (confiável), chegar a base dados do sistema. Assim, pode-se apresentar dependências de outras bases de dados que estejam em funcionamento com tais informações ou dependência direta de dados oriundos diretamente da Assembleia.

        \subsection{Limites do produto}
        O portal deve atender à legislação vigente quanto à divulgação de
        informações na Internet. Este projeto propõe uma fiscalização às
        atividades na Assembleia Legislativa, que envolve pessoas, como os
        parlamentares, assessores e secretários. As publicações no portal,
        inevitavelmente, devem envolver tais atores, desde que os mesmos
        participem de alguma atividade caracterizada por seu aspecto público ou
        que envolva, de algum modo, o uso de recursos públicos. Não serão
        publicadas, portanto, informações de cunho estritamente pessoal.

    \section{Requisitos funcionais do produto}
    \begin{itemize}
    \item Ter uma página com o perfil de cada parlamentar: considera-se este
        requisito como um dos mais básicos, uma vez que o usuário deverá ter
        conhecimento sobre cada um dos seus representantes, seus partidos e
        ideologias;
    \item Mostrar o envolvimento de cada parlamentar em atividades relacionadas
        ao uso do poder legislativo, como em votações, seções, comissões,
        debates e etc
    \item Estender o requisito anterior para atividades fora da Assembleia,
        como a participação do parlamentar em causas e ações sociais
    \item Visualizar, detalhadamente, os processos sofridos por cada parlamentar. A ideia é emitir uma ficha que indique “o quão limpa” é a carreira política do deputado.
    \item Transparecer os gastos com o dinheiro público. Servirá como uma
        prestação de contas públicas que será disponibilizada ao cidadão.
        Atualmente, alguns sites já fazem isso, porém de maneira obscura. A
        ideia é que os gastos possam ser monitorados e atualizados em períodos
        regulares (como semanalmente ou mensalmente). Eles podem envolver cada
        parlamentar individualmente, através do uso de verbas de custeio ou
        para emendas parlamentares ou podem envolver a Assembleia enquanto
        instituição, por exemplo, seus gastos com pagamento de assessores,
        secretários, serviços, compras e licitações.
    \item Os gastos de cada parlamentar em campanhas eleitorais também devem
        ser monitorados. De fato, esta é uma atividade que tem a ver com a
        Assembleia Legislativa, pois apenas serão disponibilizados dados
        referentes a deputados eleitos pelo voto direto (e que, portanto, estão
        em exercício do poder legislativo), e que foram declarados à Justiça
        Eleitoral no momento do registro da candidatura.
    \item Os dados referentes ao orçamento público devem ter sua devida
        atenção. Uma vez que a proposta foi aprovada em plenário, o texto deve
        ser publicado na íntegra. Outra seção deve ser criada para destacar as
        principais cláusulas do documento, como o valor total do orçamento,
        destino das verbas e previsões.
    \item Monitorar os projetos dos parlamentares (das mais diversas naturezas)
        que estão atualmente em discussão, os que foram aprovados ou
        reprovados, e os que estão em espera pela pauta. Criar infográficos
        interativos que separe bem estas categorias, evidenciando a importância
        da cada usa. Este requisito é importante pois o cidadão terá como
        conhecer as ideias do parlamento e poderá formar suas opiniões.
    \item Dedicar uma seção para aqueles projetos que foram votados e
        aprovados, mas que nunca saíram do papel. Apresentar o usuário uma
        satisfação sobre o ocorrido.
    \item Noticiar seções, eventos, congressos e comissões que venham debater
        assuntos relevantes e de interesse geral da população.
    \item Possibilitar pesquisas rápidas em históricos, dos períodos
        legislativos passados. Manter uma base de dados para este requisito.
    \item Possibilitar o contato direto o cidadão com o
    parlamentar, através do envio de e-mail.
    \end{itemize}

    \section{Precedência e prioridades}
       

    \begin{tabu}{X[-1, r] | X | X[-1, r] | X}
            \hline
            \textbf{No.} &
            \textbf{Funcionalidade} &
            \textbf{Prioridade do cliente} &
            \textbf{Entrega} \\ \hline
            1 & Página detalhada do perfil do parlamentar & 1 & 22/04/2014 \\ \hline
            2 &Busca por seções plenárias. Listar pauta, ata, resumo, data, parlamentares presentes e etc. & 1 &   22/04/2014 \\ \hline
            3 & Mostrar detalhes de cada processo sofrido pelo parlamentar. & 1 & 22/04/2014 \\ \hline
            4 & Páginas que detalham comissões em andamento & 2 & 22/04/2014 \\ \hline
            5 & Página que detalha o gasto mensal do parlamentar & 2 & 22/04/2014 \\ \hline
            6 &Seção especial para a prestação de contas do parlamentar à justiça eleitoral  & 2 & 29/04/2014 \\ \hline
            7 &Pesquisar projetos por categoria/natureza, por data, por status ou tema. & 2 & 29/04/2014 \\ \hline
            8 & Página que detalha projetos. & 2 & 29/04/2014 \\ \hline
            9 & Página especial que associa o parlamentar a atividades desenvolvidas fora a assembleia. & 3 & 06/05/2014 \\ \hline
            10 &Pesquisa histórico de processos do parlamentar. & 3 & 06/05/2014 \\ \hline
            11  &Listar a prestação de conta mensal da assembleia & 3 & 06/05/2014 \\ \hline
            12 &Página que detalha as informações do orçamento público & 3 & 6/05/2014 \\ \hline
            13 &Download do texto na íntegra do projeto do orçamento & 4 & 13/05/2014 \\ \hline
            15 &Página que noticia comissões, seções, reuniões e ouros assuntos. & 4 & 13/05/2014 \\ \hline
            16 &Pesquisar histórico de ex-parlamentar & 4 & 13/05/2014 \\ \hline
            17 &Página de contato do cidadão com o parlamentar  & 4 & 13/05/2014 \\ \hline
        \end{tabu}

    \section{Requisitos não-funcionais do produto}
    \begin{itemize}
    \item Requisitos de usabilidade: o portal deve apresentar uma interface
        amigável interativa, a fim de proporcionar ao usuário uma fácil
        navegação e compreensão do conteúdo exposto.
    \item O termo transparência deve ser convertido em um requisito
        não-funcional. Este é o diferencial do projeto proposto. De fato, deve
        ser fácil de se encontrar a informação, sem caminhos excessivamente
        longos que possam desencorajar o usuário de procurar pela informação.
    \item Toda publicação deve atender à legislação vigente. Além disso, deve
        ser feita de maneira imparcial, isto é, sem beneficiar nenhum partido
        político ou qualquer entidade.
    \item Deve-se assegurar a veracidade das informações.
    \end{itemize}

    \section{\color{red} Restrições técnicas}

\end{document}